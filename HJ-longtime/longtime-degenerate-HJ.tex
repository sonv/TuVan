\documentclass[12pt]{article}
\usepackage[draft]{marktext} 
%% Remove draft for real article, put twocolumn for two columns
\usepackage[draft]{svmacro}
\usepackage[utf8]{inputenc}
\usepackage{lineno}
\usepackage{authblk}
\usepackage[style=alphabetic, backend=biber]{biblatex}
\addbibresource{bibliography.bib}

%% commentary bubble
\newcommand{\SV}[2][]{\sidenote[colback=green!10]{\textbf{SV\xspace #1:} #2}}

%% Title 
\title{ Convergence rate to equilibrium to a Hamilton-Jacobi equation }
%\author[1]{Co-author}
\author{Truong-Son Van}
%\affil[1]{Institute}
\affil{Carnegie Mellon University}
\date{\today}

\begin{document}

%\linenumbers
\maketitle

\section{Introduction}
We consider the following Hamilton-Jacobi equation 
\begin{equation} \label{eq:main}
    \begin{dcases}
        \partial_t u(x,t) + \frac{u_x(x,t)^2}{2} + c(x)u(x,t) = 0     \\
        u(x,0) = u_0(x) \,.
    \end{dcases}
\end{equation}

The questions at hand are that suppose $c(x) \to 0$ as $x\to \infty$, e.g., $c(x) = \frac{1}{x}$, what are the
\begin{itemize}
    \item Wellposedness of stationary solution.
    \item Rate of convergence to stationary solution.
\end{itemize}

There are papers that show exponential convergence of~\eqref{eq:main} when $c(x) \geq \gamma >0$ but nothing if $c(x)$ vanishes. In particular, suppose $c(x) \defeq 1 $, then the ergodic problem 

\begin{equation}
    \frac{\bar u_x^2}{2} + \bar u(x) = c 
\end{equation}
has a unique(?) solution.
Then, it is known that 
\begin{equation*}
    \lim_{t\to\infty} \abs{ u(x,t) - \bar u(x)} \leq C_1e^{-C_2t} \,.
\end{equation*}
See~\cite{FujitaLoreti2009}.

\section{Ergodic problem}
We consider the ergodic problem 
\begin{equation}\label{eq:ergodic}
    \tag{E}
    \frac{u}{x} + H(u_x) = c \qquad \text{for } x\in \R\,.
\end{equation}

We study the wellposedness of  this equation by the method of vanishing viscosity. 
In particular, by classical theory of elliptic equations, the equation 
\begin{equation} \label{eq:ergodic-eps}
    \tag{E$_\epsilon$}
    \frac{u}{x} + H(u_x) + \epsilon u_{xx} = 0 \quad \text{for } x \in \R 
\end{equation}
is wellposed. Thus, for each $\epsilon >0$, let $u^\epsilon$ be the unique solution of~\eqref{eq:ergodic-eps}.
\begin{lemma}
    There exists a constant $c$, a function $\bar u$ and a subsequence $\set{\epsilon_i}_{i\in \N}$ such that 
    \begin{equation}
        \frac{\bar u}{x} + H(\bar u_x) = c \,,
    \end{equation}
    and $u_{\epsilon_i} \to \bar u$ locally uniformly.
\end{lemma}

\printbibliography 
%\bibliography{refs}
%\bibliographystyle{halpha-abbrv}


\end{document}
