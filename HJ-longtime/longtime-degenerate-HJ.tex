\documentclass[12pt]{article}
\usepackage[draft]{marktext} 
%% Remove draft for real article, put twocolumn for two columns
\usepackage[draft]{svmacro}
\usepackage[utf8]{inputenc}
\usepackage{lineno}
\usepackage{authblk}
\usepackage[style=alphabetic, backend=biber]{biblatex}
\addbibresource{bibliography.bib}

%% commentary bubble
\newcommand{\SV}[2][]{\sidenote[colback=green!10]{\textbf{SV\xspace #1:} #2}}

%% Title 
\title{ Convergence rate to equilibrium to a Hamilton-Jacobi equation }
\author[1]{Son Nguyen Thai Tu}
\author[2]{Truong-Son Van}
\affil[1]{University of Wisconsin, Madison}
\affil[2]{Carnegie Mellon University}
\date{\today}

\begin{document}

%\linenumbers
\maketitle

\section{Introduction}
We consider the following Hamilton-Jacobi equation 
\begin{equation} \label{eq:main}
    \begin{dcases}
        \partial_t u(x,t) + \frac{u_x(x,t)^2}{2} + c(x)u(x,t) = 0     \\
        u(x,0) = u_0(x) \,.
    \end{dcases}
\end{equation}

The questions at hand are that suppose $c(x) \to 0$ as $x\to \infty$, e.g., $c(x) = \frac{1}{x}$, what are the
\begin{itemize}
    \item Wellposedness of stationary solution.
    \item Rate of convergence to stationary solution.
\end{itemize}

There are papers that show exponential convergence of~\eqref{eq:main} when $c(x) \geq \gamma >0$ but nothing if $c(x)$ vanishes. In particular, suppose $c(x) \defeq 1 $, then the problem 

\begin{equation}
    \bar u + \frac{\bar u_x^2}{2}  =  0
\end{equation}
has a unique solution because we can use comparison prinple here.
Then, it is known that 
\begin{equation*}
    \lim_{t\to\infty} \abs{ u(x,t) -  \bar u(x)} \leq C_1e^{-C_2t} \,.
\end{equation*}
See~\cite{FujitaLoreti2009}. The proof of this boils down to constructing supersolution, which is not hard.

\section{Stationary problem}
We consider the ergodic problem 
\begin{equation}\label{eq:ergodic}
    \tag{E}
    \frac{\bar u}{x} + H(\bar u_x) = 0 \qquad \text{for } x\in \R\,.
\end{equation}
This equation when $H(p) = p^2$ has two smooth solutions, $\bar u = x$ and $\bar u = 0$.
Thus, to study wellposedness, we ought to impose more condition, e.g. sublinear property.
To this end, we can use the cut-off trick in the paper~\cite{TranVan2019}.


\printbibliography 
%\bibliography{refs}
%\bibliographystyle{halpha-abbrv}


\end{document}
